\documentclass[Article, letterpaper,12pt]{article}
\usepackage[left=2cm,top=2.5cm,right=2cm,bottom=3cm]{geometry} 
\usepackage[utf8]{inputenc}
\usepackage{tikz}
\usepackage{enumerate}%para cambiar el tipo de enumeración
\usepackage{etoolbox,fancyhdr}%para encabezado y pie de página
\usepackage[hidelinks]{hyperref}%Que las secciones estén enlazadas
\usepackage{afterpage} 
\usepackage{amsmath}
\usepackage{amsfonts}
\usepackage{amssymb}
\usepackage{multicol}
\usepackage[spanish]{babel}
\usepackage{listings}
\usepackage{color}
\usepackage{float}
\usepackage[font=footnotesize,labelfont=bf,labelsep=period]{caption}

\usepackage{comment}
\usepackage[backend=biber,style=authoryear]{biblatex}
\addbibresource{biblio.bib}

\lstset{language=Matlab,%
basicstyle=\small\ttfamily,
breaklines=true,%
morekeywords={matlab2tikz},
keywordstyle=\color{blue},%
morekeywords=[2]{1}, keywordstyle=[2]{\color{black}},
identifierstyle=\color{black},%
stringstyle=\color{mylilas},
commentstyle=\color{mygreen},%
showstringspaces=false,%without this there will be a symbol in the places where there is a space
%Frame
frame=tb,
captionpos=t,
xleftmargin=1em,
numbersep=0.3em,
numbers=left,
framexleftmargin=1.1em,
framexrightmargin=0pt,
emph=[1]{for,end,break},emphstyle=[1]\color{red}, %some words to emphasise
%emph=[2]{word1,word2}, emphstyle=[2]{style},    
}

\renewcommand{\lstlistingname}{Código}% Listing -> Código
\DeclareCaptionFormat{listing}{\rule{\dimexpr\linewidth\relax}{0.4pt}\par\vskip1pt#1#2#3}
\captionsetup[lstlisting]{format=listing,singlelinecheck=false, margin=0pt,position=bottom}


\definecolor{griso}{gray}{0.2}
\definecolor{grisc}{gray}{0.8}
\definecolor{grism}{gray}{0.9}
\definecolor{shadecolor}{rgb}{1,0.8,0.3}
\definecolor{aqua}{rgb}{0.0, 1.0, 1.0}
\definecolor{aliceblue}{rgb}{0.94, 0.97, 1.0}
\definecolor{aquamarine}{rgb}{0.5, 1.0, 0.83}
\definecolor{beaublue}{rgb}{0.74, 0.83, 0.9}
\definecolor{asparagus}{rgb}{0.53, 0.66, 0.42}
\definecolor{aquamarine}{rgb}{0.5, 1.0, 0.83}
\definecolor{mygreen}{RGB}{28,172,0} 
\definecolor{mylilas}{RGB}{170,55,241}


\author{totito del boom}

\renewcommand{\footrulewidth}{0.5pt}

    \pagestyle{fancy}
    \headheight=1cm
    \rhead{\bfseries}
    \fancyhead[L]{
    \begin{minipage}{3cm}
    \includegraphics[width=1.5\textwidth]{logo_udp}
    \end{minipage}}

    \fancyhead[R]{\begin{minipage}{2.2cm}
   	\large{EIT-UDP}
    \end{minipage}}

    \fancyhead[C]{
        \begin{minipage}{3cm}
            \large{Tarea 2}
        \end{minipage}}

    \fancyfoot[R]{
    Página \thepage}

    \fancyfoot[c]{}

    \fancyfoot[L]{
   C.N}

\begin{document}
\sloppy

\begin{titlepage}

\begin{center}
\vspace*{-1in}
\begin{figure}[htb]
\begin{center}
\includegraphics[width=15cm]{udplogo.png}
\end{center}
\end{figure}


ESCUELA DE INGENIERÍA EN INFORMÁTICA Y TELECOMUNICACIONES\\
\vspace*{0.25in}
FACULTAD DE INGENIERÍA Y CIENCIAS\\
\vspace*{0.25in}

\vspace*{0.3in}
\begin{large}
\begin{Huge}
 Tarea 2
\end{Huge}
\vspace{10pt}
   
\end{large}

\vspace*{0.4in}
\begin{large}
\begin{center}
\begin{huge}
$2019$
\end{huge}
\end{center}
\end{large}
\vspace*{0.4in}
\begin{large}
 Cristóbal Nuñez$^1$
\\
\begin{small}
$1$: Ingeniería Civil en Informática y Telecomunicaciones.\\



{\color{blue}
\href{mailto:cristobal.nunez@mail.udp.cl}{cristobal.nunez@mail.udp.cl}\\

}




\end{small}\\ 
\end{large}
\vspace*{0.4mm}
\rule{90mm}{0.1mm}\\
\vspace*{0.2in}
    \begin{large}
        Profesor: Martín Gutierrez\\
    \end{large}
\end{center}


\newpage 
\tableofcontents

\newpage
\end{titlepage}

\section{Introducción}
Radio Mobile es un software que permite al usuario poder utilizar datos extraídos de la NASA para lograr realizar de la manera más precisa un diseño topográfico y expresar simulaciones de Redes Móviles. Junto a Google Earth el siguiente informe consistirá en realizar una Red Móvil 4G en una zona geográfica acotada, correspondiente a la comuna de San Clemente, VII Región.
El fin de este informe es lograr responde un listado de preguntas que no permitiré comprender de la mejor forma la manera en como se desarrolla un proyecto de Redes Móviles aportando trabajando con valores óptimos y eficientes.

\section{Marco Teórico}
Para realizar esta experiencia se requirieron los conocimientos sobre redes ligadas LTE obtenidos de las cátedras y las clases, la siguiente tabla nos ayudará a comprender de mejor manera los resultados para lograr saber como afecta el ancho de banda en las comunicaciones Móviles, debemos recordar que estos resultados son en una modulación de \textit{16QAM}.

Las \textit{redes LTE} viajan en un  canal físico de transmisión de información se llamado resource block (RB), el cual esta compuesto de 12 subportadoras y 7 slots de tiempo (0.5 ms). La combinación de 2 RB se conoce como \textit{physical resource block - PRB} y es el bloque de recurso
que se entrega en la interfaz aire. La combinación de una sub-portadora con un slot
se llama resource element (RE), de esta forma un RB posee \textit{84 RE (12 subcarriers
x 7 slots).
}

\begin{figure}[H]
    \centering
    \includegraphics{AnchoBanda.PNG}
    \caption{Imagen MIMO}
    \label{fig:my_label}
\end{figure}


\section{Desarrollo}
Para comenzar a generar y diseñar esta \textit{Red LTE} se requirió de un amplio trabajo en el entendimiento del Software dado que hubo problemas en generar las redes debido a la mala y errónea implementación del archivo\textit{ .map}, por lo que generaba errores.

El diseño que implementaremos en nuestra Red LTE se basa en que los enlaces estan en la frecuencia de lo 1900(mientras que para bandas altas una ganancia de 18 dB es el valor mas común.) \textit{MHz} con un ancho de banda de 15 \textit{MHz}, con un total de 75 RBs debido a como 1 RB posee 12 subcarriers, entonces en los 15MHz alcanzamos ese número de 75 RBs.
También colocamos en los valores en la antena de DL 46 dBm por puerto, con una ganancia de 18 dBM y en el UL 23 dBm de potencia por puerto con una ganancia de 1 dBm (Recordemos que cada UE utiliza 4 RBs)


En el primer ejemplo para comenzar a tener los valores, colocamos una tasa de velocidad de 1 Mbps en DL y 256 Kbps en UL, con un CQI de 12 que tiene 6 bits funcionando con una modulación  de 64 QAM y un ancho de banda de 15 MHz con MIMO 2x2 (2x40w) en la banda de los
1,9 GHz con una morfología urbana que tiene una magnitud de 8,8 decibeles, también debemos comprender que todos estos datos son para obtener los valores que necesitamos y calcular el \textit{pathloss}  completo.
El \textit{bodyloss} será de 0 dB.
La perdida Perdidas por \textit{inbuilding} al ser urbano toma valores de 15[dB]

Lo primero fue haber generado el mapa utilizando las siguientes propiedades del Mapa:

\begin{figure}[H]
    \centering
    \includegraphics[width=8cm, height=8cm]{MapaPropiedades.PNG}
    \caption{Propiedades de mapa}
    \label{fig:my_label}
\end{figure}

Luego de Haber realizado esta operación se contempla a Realizar el diseño completo de la red. Esta Red será inicializada en la frecuencia de 1900 MHz con un ancho de banda de 15 MHz. Luego tendrá todas las características que aparecen en la foto.

\begin{figure}[H]
    \centering
    \includegraphics[width=8cm, height=8cm]{Redes.PNG}
    \caption{Propiedades de Red}
    \label{fig:my_label}
\end{figure}

Para lograr ingresar valores a las redes, se debe abrir la pestaña de miembros para lograr generar los enlaces entre los distintos terminales con respecto de los nodos.
\begin{figure}[H]
    \centering
    \includegraphics[width=8cm, height=8cm]{Miembros.PNG}
    \caption{Propiedades de la Red}
    \label{fig:my_label}
\end{figure}

Esta es la imagen final que se logra generar luego de haber implementado el mapa en el una archivo .map, este archivo tendrá sobre este las redes y los enlaces dibujados.

\begin{figure}[H]
    \centering
    \includegraphics[width=8cm, height=8cm]{Mapa.PNG}
    \caption{Sistemas de las propiedades de Red}
    \label{fig:my_label}
\end{figure}

Luego de haber diseñado la Red y haber colocado todos los valores predeterminados, se da inicio al análisis utilizando la cobertura de radio polar y el enlace de radio para ver y calcular los Link Budget.

\section{Resultados}
 Los resultado en el diseño de la red móviles fueron haber obtenido un umbral de corte necesario para cumplir los requerimientos de velocidad de un usuario de  -94,38547521 dBM en el borde de celda, teniendo en cuenta que porcentaje de PRB que se le asignará al usuario es de un 3\% debido a que ocupa 4 RB de los 75 totales  con el fin de cumplir los requerimientos de velocidad.
 El Cell Radius tomó un valor de 1,038979167 indoor con una superficie de cobertura de 3,391287175 $km^{2}$. Todo los valores descritos son en el ambiente \textit{Indoor}, si queremos analizar el ambiente \textit{Outdoor} que posee un umbral de corte de -79,38547521


\section{Preguntas}
Para comprender de la mejor manera este informe, se responderán unas cuantas preguntas que nos ayuda salir de toda duda y problemáticas de las Redes Móviles y sus dependencias.


\subsection{Pregunta 1}
En base a lo descrito anteriormente, si se modificara la portadora de la señal 4G a una frecuencia de 700 MHz, ¿Existirían
cambios en el umbral de corte? ¿Cambiaría alguno de los resultados obtenidos en el Link Budget? Justifique su respuesta.

Si se fija en las siguiente imágenes se logra observar el enlace de radio de una terminal con un nodo en la Red a una distancia de 1,02 Kilómetros, utilizando las frecuencias de 1900 MHz y otra de 700 MHz.

\begin{figure}[H]
    \centering
    \includegraphics[width=10cm, height=10cm]{Enlace16.PNG}
    \caption{Sistemas de las propiedades de Red}
    \label{fig:my_label}
\end{figure}

\begin{figure}[H]
    \centering
    \includegraphics[width=10cm, height=10cm]{Terminal16700.PNG}
    \caption{Sistemas de las propiedades de Red}
    \label{fig:my_label}
\end{figure}

Como se logra percibir ambos gráficos son distintos, por lo que implica significativamente en que si existen cambios al variar las frecuencias, esto es debido a que en la modulación digital, el espectro de banda influye directamente en la velocidad de transmisión y recepción. A los 1900 MHz se obtiene una señal con nivel Rx de 39,6 dBm, en cambio utilizando el espectro de banda se obtiene un resultado de 31,9 dBm.

Utilizando la tabla de excel obteniendo nuevos valores, a 700MHz tendríamos una variación en la ganancia de la antena y perdidas por estar en Inbuilding. Llegando a los valores de un umbral de corte de -108,1772877dBM, un cell 
Radious de 3,550278321 Km con una superficie de cobertura de 39,9 Km
39,59822229

\subsection{Pregunta 2}
Si ahora se modificara la morfología a denso urbano, ¿Existirían cambios en el umbral de corte? ¿Cambiaría alguno de los
resultados obtenidos en el Link Budget? Justifique su respuesta.

\begin{figure}[H]
    \centering
    \includegraphics[width=10cm, height=10cm]{Sincity.PNG}
    \caption{Sistemas de las propiedades de Red}
    \label{fig:my_label}
\end{figure}
Modificando la densidad urbana del sector hará que las antenas tengas parámetros más altos de perdida provocando que el umbral de corte disminuya y tenga un valor de -108,1772877, un cells radious de 0,4362282086 Km y una cobertura de  0,5978309291 $Km^{2}$

\subsection{Pregunta 3}

Si ahora se modificara el tamaño de la portadora a 10 MHz, ¿Existirían cambios en el umbral de corte? ¿Cambiaría alguno
de los resultados obtenidos en el Link Budget? Justifique su respuesta.

\begin{figure}[H]
    \centering
    \includegraphics[width=10cm, height=10cm]{Banda10.PNG}
    \caption{Sistemas de las propiedades de Red}
    \label{fig:my_label}
\end{figure}

Si se llegara a cambiar el ancho de banda de la portadora afectaría directamente a la cantidad de RBs del canal, por lo cual afecta también de inmediato el valor de el umbral de corte siendo de -109,1772877 dBm y un \textit{cell radius} de 2,820086311 Km  con una superficie de cobertura de 24,98478918 $Km^{2}$.

\subsection{Pregunta 4}
Si ahora se modificara la cantidad de puertos en el eNodeB a 2Tx/2Rx (MIMO 2x2 y 2 way diversity) ¿Existirían cambios en el
umbral de corte? ¿Cambiaría alguno de los resultados obtenidos en el Link Budget? Justifique su respuesta.

Según la siguiente imagen se logra percibir la forma en como afecta la cantidad de canales de transmisión y recepción en una comunicación de red móvil. Este gráfico nos ayuda a analizar por ejemplo si una comunicación utilizando cualquiera de las modulaciones digitales conocidas con distintos MIMO, trabaja a una ganancia constante ya sea de 8 dBm, el el \textit{BER MIMO 2Tx/4Rx} es menor al \textit{BER MIMO 2Tx/2Rx} concluyendo que si existirían cambios en el umbral de corte, debido a que si cambios a cantidad de puertos Rx, habrían muchas más probabilidades de errores en la comunicación.

\begin{figure}[H]
    \centering
    \includegraphics[width=10cm, height=10cm]{MIMOtxrx.png}
    \caption{Sistemas de las propiedades de Red}
    \label{fig:my_label}
\end{figure}


\subsection{Pregunta 5}
Si ahora se modificaran los diferentes esquemas de modulación (QPSK, 16QAM, 64QAM) ¿Existirían cambios en el umbral
de corte? ¿Cambiaría alguno de los resultados obtenidos en el Link Budget? Justifique su respuesta.

Si se llegara a cambiar la configuración y los esquemas de modulación desde \textit{QPSK} a \textit{16 QAM} o a \textit{64 QAM}, esta efecta directamente en la velocidad. Como lo estudiamos en clases al mandar un mensaje utilizando modulaciones de \textit{64 QAM} este utiliza velocidades de hasta \textit{100 Mbps} utilizando 6 bits por símbolo, entonces al momento de disminuir la modulación afectaría en la velocidad de envío de un mensaje debido a que tendrá menos bit para insertar símbolos.

Debemos recordar que toda esta transmisión de información es dependiente del calculo utilizando los RB y el ancho de banda del canal, apoyándonos en la tabla de la imagen 1.
Obtuvimos distintos resultado utilizando las distintas modulaciones, dado que en QPSK utiliza 2 bits para la codificación y 16 QAM utiliza 4 bits para la codificación.
En 16 QAM se logra un Umbral de corte indoor -92,02456262 de dBm, cell radius indoor de 0,3233804386 [Km] y una superficie de cobertura	de 0,3285325312 $Km^{2}$

En 16 QAM se logra un Umbral de corte indoor -100,4060751 de dBm, cell radius indoor de 1,038979167 [Km] y una superficie de cobertura	de 3,391287175 $Km^{2}$





\section{Conclusiones}

Utilizando el software RadioMobile, GoogleEarth y la planilla de calculo de Excel se ha llegado a obtener resultados, los más fundamentales son que en un canal de comunicación de una red móvil, se deben tener en consideración un montón de variables, tal al punto como la Banda por la que se transmiten los mensajes, el ancho de banda que almacena todos los RBs la densidad y el espacio físico donde se realizan todas las comunicaciones, etc.
Dado que se logró realizar y diseñar una med movil en la localidad de San Clemente, Se permitirá confeccionar una red móvil en cualquier sector del planeta tierra que tenga acceso a cobertura y conexión, pero también debemos de tener claro las variables y los parámetros. También gracias a este trabajo logro entender porque las empresas más grandes de telecomunicaciones estaban teniendo problemas en el trabajo y en la pelea del espectro electromagnético de 850 MHz, debido a que cada espectro influye directamente en el Link Budget, y conocer la importancia de ese espectro se tendría que realizar una simulación. 
Las redes móviles son sumamente importante hoy en día sabiendo que existe el IOT, todas las personas ocupan celular para comunicarse e informarse, por estos motivos saber de ellas es importantísimo, debido a que pasa un tráfico de datos sobre esta, siendo sumamente importante al momento de saber la capacidad de una Red y cuantos usuarios pueden estas conectados en ese canal.

\end{document}